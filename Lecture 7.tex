\newpage
\section{LECTURE 7}

\section{Agenda}
\begin{itemize}
    \item LQR as a QP
    \item Riccati Recursion
\end{itemize}

\section{The LQR Problem}
Consider the LQR problem - 
\begin{align}
    \min_{x_{1:n}, u_{1:n-1}} J &= \Big[ \sum_{n=1}^{N-1} \frac{1}{2} x_n^T Q_n x_n + \frac{1}{2} u_n^T R_n u_n \Big] + \frac{1}{2} x_N^T Q_N x_N
    \ &\ni \ x_{n+1} = A_n x_n + B_n u_n
\end{align}
Remember, we have that $Q\geq0$ and $R>0$. 

\subsection{Framing LQR as a QP}
Let us assume the initial state $x_1$ is given (and is not a decision variable). Define $z$: 
\begin{align}
    z = \begin{bmatrix}
       u_1 \\
       x_2 \\
       u_2 \\
       . \\
       . \\
       x_N
    \end{bmatrix}
\end{align}
Also define $H$:
\begin{align}
    H = \begin{bmatrix}
        R_1 & 0   & ... & 0 \\
        0   & Q_2 & ... & 0 \\
        \   & \   & .   & \ \\
        0   & 0   & ... & Q_N
    \end{bmatrix}
\end{align}
such that $J = \frac{1}{2} z^T H z$. We also define $C$ and $d$: 
\begin{align}
    C &= \begin{bmatrix}
        B_1 & (-I)  & ...   & ...   & ...   & 0 \\
        0   & A     & B     & (-I)  & ...   & 0 \\
        \   & \     & .     & \ \\
        0   & 0     & ...   & A_{N-1} & B_{N-1} & (-I)
    \end{bmatrix} \\ 
    d &= \begin{bmatrix}
        -A_1 x_1 \\
        0 \\
        . \\ 
        0 
    \end{bmatrix}
\end{align}
such that $C z = d$. We now have the LQR problem can be written as a standard QP:
\begin{align}
    &\min_{z} \frac{1}{2} z^T H z \\
    \ &\ni C z = d
\end{align}
The Lagrangian of this QP is: 
\begin{align}
    L(z, \lambda) = \frac{1}{2} z^T H z + \lambda^T \big[ C z - d\big]
\end{align}
and the KKT conditions are: 
\begin{align}
    \nabla_z L &= H z + C^T \lambda = 0 \\
    \nabla_{\lambda} L &= C z - d = 0
\end{align}
This may be written as:
\begin{align}
    \begin{bmatrix}
        H & C^T \\ 
        C & 0 
    \end{bmatrix}
    \begin{bmatrix}
        z \\
        \lambda
    \end{bmatrix}
    &= 
    \begin{bmatrix}
        0 \\
        d
    \end{bmatrix}
\end{align}
We get the exact answer by solving this one linear system! 

\subsection{Example}
Remember to check out the example on the double integrator / sliding brick on ice. We can compare shooting to the QP solution of the LQR problem. 

\subsection{A closer look at the LQR QP}
The QP KKT system is very sparse (i.e. lots of zeros in the constituent matrices), and has a lot of structure: 

% \documentclass{article}

\newcommand\aug{\fboxsep=-\fboxrule\!\!\!\fbox{\strut}\!\!\!}
\begin{align}
    \begin{bmatrix}
        R   & \     & \     & \     & \     &       & . & B^T &  &  \\
        \   & Q     & \     & \     & \     &       & . & -I & A^T &  \\
        \   & \     & R     & \     & \     &       & . & & B^T & \\
        \   & \     & \     & Q     & \     &       & . & & -I    & A^T \\
        \   & \     & \     & \     & R     &       & . & &   & B^T \\
        \   & \     & \     & \     & \     & Q_N   & . &   &  & -I \\
        .   & .     & .     & .     & .     & . & . &  .    &  .   & . \\
        B   & -I     & \     & \     & \     & \   & . &  0 & 0 & 0 \\
        \   & A     & B     & -I     & \     & \   & . &  0 & 0 & 0 \\
        \   & \     & \     & A     & B     & -I   & . &  0 & 0 & 0 \\
    \end{bmatrix}
    \begin{bmatrix}
        u_1 \\
        x_2 \\
        u_2 \\
        x_3 \\
        u_3 \\
        x_4 \\
        \lambda_2 \\
        \lambda_3 \\
        \lambda_4 
    \end{bmatrix}
    =
    \begin{bmatrix}
        0 \\
        0 \\
        0 \\
        0 \\
        0 \\
        0 \\
        -A x_1 \\
        0 \\
        0
    \end{bmatrix}    
\end{align}
-
\noindent
Consider the last line of upper block of this system (i.e. above the dotted line): 
\begin{align}
Q_N x_4 - \lambda_4 &= 0 \\
\implies \lambda_4 &= Q_N x_4 
\end{align}

\noindent
Now consider the second last line of the upper block of this system:
\begin{align}
    R u_3 + B^T \lambda_4 &= R u_3 + B^T Q_N x_4 = 0 
\end{align}
Plugging the dynamics into this, we have: 
\begin{align}
    R u_3 + B^T Q_N (A x_3 + B u_3) &= 0 \\
    \implies u_3 &= - (R + B^T Q_N B )^{-1} B^T Q_N A x_3
\end{align}
We can define $K_3 = (R + B^T Q_N B )^{-1} B^T Q_N A$ in the above equation. 

\noindent
Finally, consider the third line of this system. 
\begin{align}
    Q x_3 - \lambda_3 + A^T \lambda_4 = 0
\end{align}
We can then manipulate this by first substituting $\lambda_4$, then plugging in the dynamics for $x_4$, and finally using the feedback law:
\begin{align}
    Q x_3 - \lambda_3 + A^T Q_N x_4 &= 0 \\
    \implies Q x_3 - \lambda_3 + A^T Q_N (A x_3 + B u_3) &= 0 \\
    \implies Q x_3 - \lambda_3 + A^T Q_N (A - B K) x_3 &= 0 \\
    \implies \lambda_3 = (Q + A^T Q_N (A- BK)) x_3
\end{align}
We can then define $P_3 =  (Q + A^T Q_N (A- BK))$. 

\noindent
We now have a recursion for $K$ and $P$: 
\begin{align}
    P_N &= Q_N \\
    K_n &= (R + B^T P_{n+1} B)^{-1} B^T P_{n+1} A \\
    P_n &= Q + A^T P_{n+1} (A - BK_n)
\end{align}

\begin{itemize}
    \item This is called a Riccati equation / recursion.
    \itm We can solve the QP by doing a backward Riccati recursion followed by a forward rollout starting from $x_1$.
    \item This has complexity $\mathcal{O} (N (n+m)^3$ instead of $\mathcal{O} (N^3 (n+m)^3$ for the naive QP solution. 
    This carries over to the general non-linear case. 
    \item Even more importantly, we now have a feedback policy instead of an open loop trajectory to execute. 
\end{itemize}

\subsection{Example}
Remember to check out the lecture video for an example on LQR with Riccati open-loop, and closed loop with noise. 

\subsection{Infinite Horizon}
Let's now consider the infinite horizon case.
\begin{itemize}
    \item For the time-invariant LQR, $K$ matrices converge to constant values over the infinite horizon. 
    \item For stabilization problems we almost always use the constant $K$.
    \item We can solve for this explicitly in Julia / Matlab.
\end{itemize}

\section{Controllability}
How do we know if LQR will work? For the time-invariant case, there is a simple answer. For any initial state $x_0$, $x_n$ is given by: 
\begin{align}
    x_n &= A x_{n-1} + B u_{n-1} \\
    &= A (Ax_{n-2} + B u_{n-2}) + B u_{n-1} \\
    &= A^n x_0 + A_{n-1} B u_0 + ... + B u_{n-1}\\
    &= \begin{bmatrix}
        B \ AB \ A^2 B \ ... \ A^{n-1} B 
    \end{bmatrix} 
    \begin{bmatrix}
        u_{n-1} \\
        u_{n-2} \\
        . \\
        . \\
        u_0
    \end{bmatrix}
    + 
    A^n x_0
\end{align}
Here, we may define a controllability matrix $C$ as: 
\begin{align}
    C = \begin{bmatrix}
        B \ AB \ A^2 B \ ... \ A^{n-1} B 
    \end{bmatrix} 
\end{align}
In order to drive any $x_0$ to any desired $x_n$, the controllability matrix must have full row rank: 
\begin{align}
    rank(C) = n \\
    \ni n = dim(x)
\end{align}
This is equivalent to solving the following least-squares problem for $u_{0:n-1}$: 
\begin{align}
    \begin{bmatrix}
        u_{n-1} \\
        u_{n-2} \\
        . \\
        . \\
        u_0
    \end{bmatrix}
    &= 
    \begin{bmatrix}
        C^T (C C^T)^{-1} 
    \end{bmatrix}
    \begin{bmatrix}
        (x_n - A^n x_0)
    \end{bmatrix}
\end{align}
Here, $\big[ C^T (C C^T)^{-1} \big]$ is the pseudo-inverse of controllability $C$, and requires $C C^T$ to be invertible. 
We can stop at $n$ time steps because the Cayley-Hamilton theorem says that $A^n$ can e written as a linear combination of smaller powers of $A$: 
\begin{align}
    A^n = \sum_{k=0}^{n-1} \alpha_k A^k 
\end{align}
for some $\alpha_k$.
Therefore adding more timesteps / columns to $C$ can't increase the rank of $C$. 