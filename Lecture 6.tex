\newpage
\section{LECTURE 6}

\section{Agenda}
This lecture covers Pontryagin's Minimum Principle, and an introduction to Linear Quadratic Regulators. 

\section{Pontryagin's Minimum Principle}
\begin{itemize}
    \item The Pontryagin's Minimum Principle is also a ``maximum principle'' if instead of minimizing a cost function, we are maximizing a reward function. 
    \item Essentially provides first order necessary conditions of deterministic optimal control problems. 
    \item In discrete time, it's just a special case of the KKT. 
\end{itemize}
Consider the problem we have before: 
\begin{align}
    \min_{x_{1:N}, u_{1:N-1}} \ & J(x_{1:N}, u_{1:N-1}) = \sum_{k=1}^{N-1} l(x_k, u_k) + l_F(x_N) \\
    \ni \ & x_{n+1} = f(x_n,u_n)
\end{align}
In this setting, we will consider torque limits applied to the problem, but it's hard to handle constraints on state (such as collision constraints, like we had before). 

\noindent
We can form the Lagrangian of this problem as follows: 
\begin{align}
    L = \sum_{k=1}^{N-1} \Big[ l(x_k,u_k) + \lambda_{k+1}^+ (f(x_k,u_k)-x_{k+1}) \Big] + l_F(x_N) \\
\end{align}
This result is usually stated in terms of the ``Hamiltonian'': 
\begin{align}
    H(x, u, \lambda) = l(x,u) + \lambda^T f(x,u)
\end{align}
Plugging in $H$ into the Lagrangian $L$:
\begin{align}
    L = H(x_1, u_1, \lambda_2) + \Big[ \sum_{k=2}^{N-1} H(x_k,u_k,\lambda_{k+1}) - \lambda_k^T x_k \Big] + l_F(x_N) - \lambda_N^T x_N 
\end{align}
Note the change in indexing of the summation.
If we take derivatives with respect to $x$ and $\lambda$: 
\begin{align}
    \frac{\partial L}{\partial \lambda_k} &= \frac{\partial H}{\partial \lambda_k} - x_{k+1} = f(x_k,u_k) - x_{k+1} = 0 \\
    \frac{\partial L}{\partial x_k} &= \frac{\partial H}{\partial x_k} - \lambda_{k}^T = \frac{\partial l}{\partial x_k} + \lambda_{k+1}^T \frac{\partial f}{\partial x_k} - \lambda_k^T = 0 \\
    \frac{\partial L}{\partial x_N} &= \frac{\partial l_F}{\partial x_N} - \lambda_N^T = 0
\end{align}
For $u$, we write the $\min$ explicitly to handle the torque limits:
\begin{align}
    u_k &= \arg\min_{\Tilde{u}} H(x_k, \Tilde{u}, \lambda_{k+1}) \\
    \ni \ & \Tilde{u}  \in \mathcal{U} 
\end{align}
Here, $\Tilde{u} \in \mathcal{U}$ is shorthand for ``in feasible set'', for example, $u_{\min} \leq \tilde{u} \leq u_{\max}$.\\

\noindent
In summary, we have: 
\begin{align}
    x_{k+1} &= \nabla_{\lambda} H(x_k, u_k, \lambda_{k+1}) = f(x_k, u_k) \\
    \lambda_k &= \nabla_{x} H(x_k,u_k,\lambda_{k+1}) = \nabla_x l(x_k,u_k) + (\frac{\partial f}{\partial x})^T \lambda_{k+1} \\
    u_k &= \arg\min_{\tilde{u}} H(x_k, \tilde{u}, \lambda_{k+1}) \\
    &\ \ \ni \ \tilde{u} \in \mathcal{U} \\
    \lambda_{N} &= \frac{\partial l_F}{\partial X_{N}}
\end{align}
Now these can be stated almost identically in continuous time: 
\begin{align}
    \dot{x} &= \nabla_{\lambda} H(x, u, \lambda) = f_{\rm continuous}(x, u) \\
    \dot{\lambda} &= \nabla_{x} H(x,u,\lambda) = \nabla_x l(x,u) + (\frac{\partial f_{\rm continuous}}{\partial x})^T \lambda \\
    u &= \arg\min_{\tilde{u}} H(x, \tilde{u}, \lambda) \\
    &\ \ \ni \  \tilde{u} \in \mathcal{U} \\
    \lambda_{N} &= \frac{\partial l_F}{\partial X_{N}}
\end{align}

\subsection{Notes}
\begin{itemize}
    \item Historically, many algorithms were based on forward / backward integration of the continuous ODEs for $x(t), \lambda(t)$ for performing gradient descent on $u(t)$. 
    \item Thse are called ``indirect'' and / or ``shooting'' methods. 
    \item In continuous time, $\lambda(t)$ is called ``co-state'' trajectory. 
    \item These methods have largely fallen out of favor as computers and solvers have improved. 
\end{itemize}

\section{Linear Quadratic Regulator (LQR)}
A very common class of controllers is the Linear Quadratic Regulator. In this setting, we have a quadratic cost, and linear dynamics, specified as: 
\begin{align}
    \min_{x_{1:N}, u_{1:N-1}} &\sum_{k=1}^{N-1} \Big[ \frac{1}{2} x_k^T Q x_k + \frac{1}{2} u_k^T R_k u_k  \Big] + \frac{1}{2} x_N^T Q_N x_N \\
    & \ \ni \ x_{k+1} = A_k x_k + B_k u_k 
\end{align}
Here, we have: $Q \geq 0, R > 0$. 
\begin{itemize}
    \item The goal of this problem is to drive the system to the origin.
    \item It's considered a ``time-invariant'' LQR if $A_k=A, B_k=B, Q_k=Q, R_k=R \ \forall \ k$, and is a ``time-varying'' LQR (TVLQR) otherwise. 
    \item We typically use time-invariant LQRs for stabilizing an equilibrium, and TVLQR for tracking trajectories.
    \item Can (locally) approximate many non-linear problems, and are thus very commonly used. 
    \item There are also many extensions of LQR, including the infinite horizon case, and stochastic LQR. 
    \item It's been called the ``crown jewel of control theory.''
\end{itemize}

\subsection{LQR with Indirect Shooting}
Consider: 
\begin{align}
    x_{k+1} &= A x_k + B u_k \\
    \lambda_k &= Q x_k + A^T \lambda_{k+1}, \ \lambda_N = Q x_N \\
    u_{k} &= -R^{-1} B^T \lambda_{k+1} \ \textrm{This gives the gradients of the cost w.r.t.} u_k
\end{align}

The procedure for LQR with indirect shooting is: 
\begin{enumerate}
    \item Start with an initial guess trajectory. 
    \item Simulate (or ``rollout'') to get $x(t)$.
    \item Backward pass to get $\lambda(t)$ and $\Delta u(t)$.
    \item Rollout with line search on $\Delta u$.
    \item Go to (3) until convergence.
\end{enumerate}

\subsection{Example}
Check out the example of the double integrator in the lecture. Here, we have - 
\begin{align}
    \dot{x} = \begin{bmatrix}
        \dot{q} \\
        \ddot{q} 
    \end{bmatrix}
    = 
    \begin{bmatrix}
        0 \ \ 1  \\
        0 \ \ 0
    \end{bmatrix}
    \begin{bmatrix}
        q \\
        \dot{q} 
    \end{bmatrix}
    + 
    \begin{bmatrix}
        0 \\
        1
    \end{bmatrix}
    u 
\end{align}
Think of this as a ``sliding brick'' on ice, without friction. Here, the discrete-time version of this system is - 
\begin{align}
    x_{k+1} = 
    \begin{bmatrix}
        1 \ \ h  \\
        0 \ \ 1
    \end{bmatrix}
    \begin{bmatrix}
        q_k \\
        \dot{q_k} 
    \end{bmatrix}
    + 
    \begin{bmatrix}
        \frac{1}{2} h^2 \\
        h
    \end{bmatrix}
    u_k 
\end{align} 
where $    \begin{bmatrix}
        1 \ \ h  \\
        0 \ \ 1
    \end{bmatrix}$ represents the $A$ matrix, and $    \begin{bmatrix}
        \frac{1}{2} h^2 \\
        h
    \end{bmatrix}$ represents the $B$ matrix.
    
\subsection{Bryson's Rule}
\begin{itemize}
    \item Cost-tuning heuristic. 
    \item Set diagonal elements of $Q$ and $R$ to $\frac{1}{ \textrm{max value}^2 }$.
    \item Normalizes all cost terms to $1$. 
    \item Good starting point for tuning. 
    \item In the scalar case, if we have: 
    \begin{align}
        \frac{1}{2} Q x^2 + \frac{1}{2} R u^2 
    \end{align}
    Then we can set $Q$ such that $Q x^2 = \frac{1}{x_{\max}^2 } x_{\max}^2 = 1$ , and similarly $R$ such that: $R u^2 = \frac{1}{u_{\max}^2} u_{\max}^2 = 1$
\end{itemize}
